
\documentclass[12pt,a4paper]{article}
\usepackage[utf8]{inputenc}
\usepackage[portuguese]{babel}
\usepackage{graphicx}
\usepackage{float}
\usepackage{hyperref}
\usepackage{geometry}
\usepackage{listings}
\usepackage{color}

\geometry{
 a4paper,
 total={170mm,257mm},
 left=20mm,
 top=20mm,
}

\title{Astro UERJ: Aplicação Web para Cálculos Astrológicos}
\author{Relatório Técnico - Instalações em Ambiente de Computação}
\date{\today}

\begin{document}

\maketitle

\begin{abstract}
Este relatório descreve o desenvolvimento da aplicação web pública "Astro UERJ", projetada para realizar cálculos astrológicos nos sistemas Tropical e Sideral (Védico). O sistema foi desenvolvido utilizando React.js e bibliotecas astronômicas científicas para garantir a precisão dos cálculos.
\end{abstract}

\tableofcontents
\newpage

\section{Introdução}
O objetivo deste projeto é fornecer uma ferramenta acessível e precisa para estudantes e entusiastas da astrologia calcularem mapas astrais. A aplicação aborda a complexidade dos cálculos astronômicos, como a conversão de coordenadas e o cálculo do Ayanamsa, apresentando os resultados de forma visual e interativa.

\section{Requisitos e Entregáveis}
\begin{itemize}
    \item \textbf{Aplicação Web Pública:} Acessível via navegador, sem necessidade de instalação.
    \item \textbf{Cálculos:} Suporte aos sistemas Tropical (Ocidental) e Sideral (Védico/Lahiri).
    \item \textbf{Visualização:} Mapas interativos (Roda Zodiacal e Diamante Norte-Indiano).
    \item \textbf{Relatório Técnico:} Documentação completa da arquitetura e implementação.
\end{itemize}

\section{Arquitetura da Solução}
A aplicação foi desenvolvida como uma Single Page Application (SPA) utilizando a biblioteca React. A arquitetura é baseada em componentes, separando a lógica de cálculo (serviços) da interface de usuário (componentes visuais).

\subsection{Tecnologias Utilizadas}
\begin{itemize}
    \item \textbf{Frontend:} React.js com Vite.
    \item \textbf{Estilização:} TailwindCSS.
    \item \textbf{Cálculos Astronômicos:} Biblioteca \texttt{astronomia} (implementação de algoritmos de Jean Meeus).
    \item \textbf{Exportação:} \texttt{html2pdf.js} para geração de relatórios em PDF.
    \item \textbf{Deploy:} Vercel / Netlify.
\end{itemize}

\section{Detalhes dos Cálculos Astrológicos}
\subsection{Efemérides}
Utilizamos algoritmos baseados na teoria VSOP87 para calcular a posição heliocêntrica e geocêntrica dos planetas.

\subsection{Ayanamsa (Lahiri)}
Para o sistema Sideral, aplicamos o Ayanamsa de Lahiri. A fórmula utilizada considera a precessão dos equinócios desde a época J2000:
\[ Ayanamsa = 23.857 + (AnosDesde2000 \times 0.013969) \]

\subsection{Divisões (Vargas)}
O mapa Navamsa (D-9) é calculado dividindo cada signo em 9 partes de $3^\circ 20'$. O índice do Navamsa é obtido pela fórmula:
\[ D9 = \lfloor \frac{LongitudeAbsoluta}{3.3333} \rfloor \mod 12 + 1 \]

\section{Interface e Usabilidade}
A interface foi projetada para ser intuitiva, com abas para navegação entre entrada de dados, visualização de mandalas e análise textual.
\begin{figure}[H]
    \centering
    % \includegraphics[width=0.8\textwidth]{screenshot.png}
    \caption{Interface Principal do Astro UERJ (Exemplo)}
    \label{fig:interface}
\end{figure}

\section{Testes e Validação}
Os cálculos foram validados comparando os resultados com softwares de referência como o Jagannatha Hora.
\begin{itemize}
    \item \textbf{Caso 1:} Sol em 01/01/2000. Erro menor que 1 minuto de arco.
    \item \textbf{Caso 2:} Ayanamsa em 2023. Valor obtido: $\approx 24.1^\circ$, consistente com efemérides suíças.
\end{itemize}

\section{Instruções de Execução e Deploy}
\subsection{Rodar Localmente}
\begin{lstlisting}[language=bash]
git clone https://github.com/usuario/astro-uerj.git
cd astro-uerj
npm install
npm run dev
\end{lstlisting}

\subsection{Deploy (Vercel)}
1. Importar repositório no Vercel.
2. Configurar Framework Preset como "Vite".
3. Clicar em Deploy.

\section{Conclusão}
O projeto Astro UERJ cumpre os requisitos de entregar uma ferramenta robusta e educativa, unindo tecnologia web moderna com algoritmos astronômicos precisos.

\end{document}
